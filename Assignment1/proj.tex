%iffalse
\let\negmedspace\undefined
\let\negthickspace\undefined
\documentclass[journal,12pt,twocolumn]{IEEEtran}
\usepackage{cite}
\usepackage{amsmath,amssymb,amsfonts,amsthm}
\usepackage{algorithmic}
\usepackage{graphicx}
\usepackage{textcomp}
\usepackage{xcolor}
\usepackage{txfonts}
\usepackage{listings}
\usepackage{enumitem}
\usepackage{mathtools}
\usepackage{gensymb}
\usepackage{comment}
\usepackage[breaklinks=true]{hyperref}
\usepackage{tkz-euclide} 
\usepackage{listings}
\usepackage{gvv}                                        
%\def\inputGnumericTable{}                                 
\usepackage[latin1]{inputenc}                                
\usepackage{color}                                            
\usepackage{array}                                            
\usepackage{longtable}                                       
\usepackage{calc}                                             
\usepackage{multirow}                                         
\usepackage{hhline}                                           
\usepackage{ifthen}                                           
\usepackage{lscape}
\usepackage{tabularx}
\usepackage{array}
\usepackage{float}
\usepackage{xparse}


\newtheorem{theorem}{Theorem}[section]
\newtheorem{problem}{Problem}
\newtheorem{proposition}{Proposition}[section]
\newtheorem{lemma}{Lemma}[section]
\newtheorem{corollary}[theorem]{Corollary}
\newtheorem{example}{Example}[section]
\newtheorem{definition}[problem]{Definition}
\newcommand{\BEQA}{\begin{eqnarray}}
\newcommand{\EEQA}{\end{eqnarray}}
\newcommand{\define}{\stackrel{\triangle}{=}}
\theoremstyle{remark}
\newtheorem{rem}{Remark}

% Marks the beginning of the document
\begin{document}
\bibliographystyle{IEEEtran}
\vspace{3cm}

\title{Learning {\LaTeX}}
\author{EE24BTECH11053 - S A Aravind Eswar$^{*}$}
\maketitle
\newpage
\bigskip

\renewcommand{\thefigure}{\theenumi}
\renewcommand{\thetable}{\theenumi}

\section{SECTION B}
6. If $f(1)=1,f^1(1)=2$, then $\displaystyle\lim\limits_{x\to 1} \frac{{\sqrt{f(x)}}-1}{{\sqrt{x}}-1}$ is  \hfill{[2002]}\\
\begin{enumerate}
    \item 2
    \item 4
    \item 1
    \item {$\frac{1}{2}$}\\[2pt]
\end{enumerate}


7. $f$ is defined in [-5,5] as\hfill [2002]\\
    $f(x) = x$ if $x$ is rational\\
    $\-$ $\-$ $\-$ $\-$ $\-$ $\-$ $= -x$ if $x$ is irrational. Then

\begin{enumerate}
    \item $f(x)$ is continuous at every $x$, except $x=0$
    \item $f(x)$ is discontinuous at every $x$, except $x=0$
    \item $f(x)$ is continuous everywhere
    \item $f(x)$ is discontinuous everywhere\\[2pt]
\end{enumerate}

8. $f(x)$ and $g(x)$ are two differentiable functions on [0,2] such that $f''(x)-g''(x)=0, f'(1)=2g'(1)=4,f(2)=3g(2)=9$ then $f(x)-g(x)$ at $x=\frac{3}{2}$ is\hfill{[2002]}
\begin{enumerate}
    \item 0
    \item 2
    \item 10
    \item 5\\[2pt]
\end{enumerate}

9. If $f(x+y)=f(x).f(y)\forall x,y$ and $f(5)=2,f'(0)=3$, then $f'(5)$ is \hfill [2002]

\begin{enumerate}
	\item 0
	\item 1
	\item 6
	\item 2\\[2pt]
\end{enumerate}

10. $\displaystyle\lim\limits_{x\to\infty}\frac{1+2^4+3^4+\dots n^4}{n^5}$$-$$\displaystyle\lim\limits_{x\to\infty}\frac{1+2^3+3^3+\dots n^3}{n^5}$ \hfill{[2003]}
\begin{enumerate}
    \item $\frac{1}{5}$
    \item $\frac{1}{30}$
    \item Zero
    \item $\frac{1}{4}$\\[2pt]
\end{enumerate}

11. If $\displaystyle \lim\limits_{x\to 0}\frac{\log(3+x)-\log(3-x)}{x} = k$, then the value of $k$ is \hfill{[2003]}
\begin{enumerate}
    \item $-\frac{2}{3}$
    \item $0$
    \item $-\frac{1}{3}$
    \item $\frac{2}{3}$\\[2pt]
\end{enumerate}

12. The value of $\displaystyle \lim\limits_{x\to 0}\frac{\int_{0}^{x^2} \sec^2tdt}{x\sin x}$ is \hfill [2003]
\begin{enumerate}
    \item 0 
    \item 3
    \item 2
    \item 1\\[2pt]
\end{enumerate}

13. Let $f(a)=g(a)=k$ and their nth derivatives $f^n(a),g^n(a)$ exist and are not equal for some $n$. Further if $\displaystyle \lim\limits_{x\to a}\frac{f(a)g(x)-f(a)-g(a)f(x)+f(a)}{g(x)-f(x)}=4$ then the value of $k$ is \hfill [2003]

\begin{enumerate}
    \item 0
    \item 4
    \item 2
    \item 1\\[2pt]
\end{enumerate}

14. $\displaystyle \lim\limits_{x\to\frac{\pi}{2}}\frac{\sbrak{1-tan\brak{\frac{x}{2}}}\sbrak{1-\sin x}}{ \sbrak{1+tan\brak{\frac{x}{2}}}\sbrak{\pi-2x}^3}$ is \hfill [2003]

\begin{enumerate}
    \item $\infty$
    \item $\frac{1}{8}$
    \item 0
    \item $\frac{1}{32}$\\[2pt]
\end{enumerate}

15. $\displaystyle \text{If } f(x)=
    \begin{dcases}
	    xe^{- \brak{\frac{1}{\abs{x}} + \frac{1}{x} }},&x\neq 0\\ 
        0                                    ,&x = 0
    \end{dcases}
   \text{ then f(x) is}
$ \hfill [2003]

\begin{enumerate}
    \item discontinuous every where
    \item continuous as well as differentiable for all $x$
    \item continuous for all $x$ but not differentiable at $x=0$
    \item neither differentiable not continuous at $x=0$
\end{enumerate}

16. if $\displaystyle \lim\limits_{x\to\infty}\brak{1+\frac{a}{x}+\frac{b}{x^2}}^{2x}=e^2$, then the values of $a$ and $b$, are\hfill [2004]

\begin{enumerate}
    \item $a=1$ and $b=2$
    \item $a=1 \text{and }b\in \textbf{\textit{R}}$
    \item $a\in \textbf{\textit{R}},b=2$
    \item $a\in \textbf{\textit{R}},b\in \textbf{\textit{R}}$\\[2pt]
\end{enumerate}

17. $\displaystyle f(x)=\frac{1-\tan x}{4x-\pi}, x\neq\frac{\pi}{4},x\in\sbrak{0,\frac{\pi}{4}}\text{.If }f(x)$ is continuous in $\displaystyle\sbrak{0,\frac{\pi}{2}},$\text{then }$f\brak{\frac{\pi}{4}}$\text{is}

\begin{enumerate}
    \item $-1$
    \item $\frac{1}{2}$
    \item $-\frac{1}{2}$
    \item $1$\\[2pt]
\end{enumerate}

18. $\displaystyle\lim\limits_{n\to\infty}\sbrak{\frac{1}{n^2}\sec^2\frac{1}{n^2}+\frac{2}{n^2}\sec^2\frac{4}{n^2}\dots\dots\dots\dots\frac{1}{n}sec^2 1}$ equals\hfill[2005]

\begin{enumerate}
    \item $\frac{1}{2}\sec 1$
    \item $\frac{1}{2}\cosec 1$
    \item $\tan 1$
    \item $\frac{1}{2} \tan 1$\\[2pt]
\end{enumerate}

19. Let $\alpha$ and $\beta$ be the distinct roots of $ax^2+bx+c=0$ , then, $\displaystyle\lim\limits_{x\to\alpha}\frac{1-\cos(ax^2+bx+c)}{(x-a)^2}$ is equal to\hfill[2005]

\begin{enumerate}
    \item $\frac{a^2}{2}(\alpha-\beta)^2$
    \item $0$
    \item $\frac{-a^2}{2}(\alpha-\beta)^2$
    \item $\frac{1}{2}(\alpha-\beta)^2$\\[2pt]
\end{enumerate}

20. Suppose $f(x)$ is a differentiable at $x=1$ and $\lim\limits_{h\to 0}\frac{1}{h}f(1+h)=5$, then $f'(1)$ equals\hfill[2005]

\begin{enumerate}
    \item 3
    \item 4
    \item 5
    \item 6
\end{enumerate}
 
\end{document}

