%iffalse
\let\negmedspace\undefined
\let\negthickspace\undefined
\documentclass[journal,12pt,twocolumn]{IEEEtran}
\usepackage{cite}
\usepackage{amsmath,amssymb,amsfonts,amsthm}
\usepackage{algorithmic}
\usepackage{graphicx}
\usepackage{textcomp}
\usepackage{xcolor}
\usepackage{txfonts}
\usepackage{listings}
\usepackage{enumitem}
\usepackage{mathtools}
\usepackage{gensymb}
\usepackage{comment}
\usepackage[breaklinks=true]{hyperref}
\usepackage{tkz-euclide} 
\usepackage{listings}
\usepackage{gvv}                                        
%\def\inputGnumericTable{}                                 
\usepackage[latin1]{inputenc}                                
\usepackage{color}                                            
\usepackage{array}                                            
\usepackage{longtable}                                       
\usepackage{calc}                                             
\usepackage{multirow}                                         
\usepackage{hhline}                                           
\usepackage{ifthen}                                           
\usepackage{lscape}
\usepackage{tabularx}
\usepackage{array}
\usepackage{float}


\newtheorem{theorem}{Theorem}[section]
\newtheorem{problem}{Problem}
\newtheorem{proposition}{Proposition}[section]
\newtheorem{lemma}{Lemma}[section]
\newtheorem{corollary}[theorem]{Corollary}
\newtheorem{example}{Example}[section]
\newtheorem{definition}[problem]{Definition}
\newcommand{\BEQA}{\begin{eqnarray}}
\newcommand{\EEQA}{\end{eqnarray}}
\newcommand{\define}{\stackrel{\triangle}{=}}
\theoremstyle{remark}
\newtheorem{rem}{Remark}

% Marks the beginning of the document
\begin{document}
\bibliographystyle{IEEEtran}
\vspace{3cm}

\title{Learning {\LaTeX}}
\author{EE24BTECH11053 - S A Aravind Eswar$^{*}$}
\maketitle
\newpage
\bigskip

\renewcommand{\thefigure}{\theenumi}
\renewcommand{\thetable}{\theenumi}
\section{}
14. Let S be set of all column matrix $\begin{bmatrix}
    b_1\\
    b_2\\
    b_3
\end{bmatrix}$ such that $b_1, b_2, b_3 \in \mathbb{R}$ and the system of equations (in real variables) $$
    -x+2y+5z=b_1\\
    2x-4y+3z=b_2\\
    x-2y+2z=b_3
$$ has at least one solution. Then, which of the following system(s) (in real variables) has (have) at least one solution for each $\begin{bmatrix}
    b_1\\
    b_2\\
    b_3
\end{bmatrix}\in S$ \hfill(JEE Adv. 2018)

\begin{enumerate}
    \item $x+2y+3z=b_1, 4y+5z=b_2 \text{ and }x+2y+6z=b_3$
    \item $x+y+3z=b_1, 5x+2y+6z=b_2\text{ and }-2x-y-3z=b_3$
    \item $-x+2y-5z=b_1,2x-4y+10z=b_2\text{ and }x-2y+5z=b_3$
    \item $sx+2y+5z=b_1,2x+3z=b_2,x+4y-5z=b_3$\\[2pt]
\end{enumerate}

15. Let $M=\begin{bmatrix}
    0 & 1 & a\\
    1 & 2 & 3\\
    3 & b & 1
\end{bmatrix}$ and $(\mathop{adj}M)=\begin{bmatrix}
    -1 & 1 & -1\\
    8 & -6 & 2\\
    -5 & 3 & 1
\end{bmatrix}$ where $a$ and $b$ are real numbers. Which of the following options is/are correct? \hfill (JEE Adv. 2019)

\begin{enumerate}
    \item $a+b=3$
    \item $\mathop{det}(\mathop{adj}M^2)=81$
    \item $(adjM)^{-1}+adjM^{-1}=-M$
    \item $If M\begin{bmatrix}
        \alpha\\
        \beta\\
        \gamma
    \end{bmatrix} = \begin{bmatrix}
        1\\
        2\\
        3
    \end{bmatrix}$ then $\alpha-\beta+\gamma=3$\\[2pt]
\end{enumerate}

16. Let\\
$
P_1=I=\begin{bmatrix}
    1 & 0 & 0\\
    0 & 1 & 0\\
    0 & 0 & 1
\end{bmatrix}, P_2 = \begin{bmatrix}
    1 & 0 & 0\\
    0 & 0 & 1\\
    0 & 1 & 1
\end{bmatrix}, P_3 = \begin{bmatrix}
    0 & 1 & 0\\
    1 & 0 & 0\\
    0 & 0 & 1
\end{bmatrix}, P_4 = \begin{bmatrix}
    0 & 1 & 0\\
    0 & 0 & 1\\
    1 & 0 & 0
\end{bmatrix}, P_5 = \begin{bmatrix}
    0 & 0 & 1\\
    1 & 0 & 0\\
    0 & 1 & 0
\end{bmatrix}, P_6 = \begin{bmatrix}
    0 & 0 & 1\\
    0 & 1 & 0\\
    1 & 0 & 1
\end{bmatrix}
$\\ and $\displaystyle X = \sum_{k=1}^{6}P_k\begin{bmatrix}
    2 & 1 & 3\\
    1 & 0 & 2\\
    3 & 2 & 1
\end{bmatrix}{P_k}^T$\\Where ${P_k}^T$ denotes the transpose of matrix $P_k$. Then which of the following options is/are correct? \hfill (JEE Adv. 2019)

\end{document}

