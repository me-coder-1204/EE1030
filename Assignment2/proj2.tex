%iffalse
\let\negmedspace\undefined
\let\negthickspace\undefined
\documentclass[journal,12pt,twocolumn]{IEEEtran}
\usepackage{cite}
\usepackage{amsmath,amssymb,amsfonts,amsthm}
\usepackage{algorithmic}
\usepackage{graphicx}
\usepackage{textcomp}
\usepackage{xcolor}
\usepackage{txfonts}
\usepackage{listings}
\usepackage{enumitem}
\usepackage{mathtools}
\usepackage{gensymb}
\usepackage{comment}
\usepackage[breaklinks=true]{hyperref}
\usepackage{tkz-euclide} 
\usepackage{listings}
\usepackage{gvv}                                        
%\def\inputGnumericTable{}                                 
\usepackage[latin1]{inputenc}                                
\usepackage{color}                                            
\usepackage{array}                                            
\usepackage{longtable}                                       
\usepackage{calc}                                             
\usepackage{multirow}                                         
\usepackage{hhline}                                           
\usepackage{ifthen}                                           
\usepackage{lscape}
\usepackage{tabularx}
\usepackage{array}
\usepackage{float}
\usepackage{xparse}

\newcommand\Perm[2][^n]{\prescript{#1\mkern-2.5mu}{}P_{#2}}
\newcommand\Comb[2][^n]{\prescript{#1\mkern-0.5mu}{}C_{#2}}

\newtheorem{theorem}{Theorem}[section]
\newtheorem{problem}{Problem}
\newtheorem{proposition}{Proposition}[section]
\newtheorem{lemma}{Lemma}[section]
\newtheorem{corollary}[theorem]{Corollary}
\newtheorem{example}{Example}[section]
\newtheorem{definition}[problem]{Definition}
\newcommand{\BEQA}{\begin{eqnarray}}
\newcommand{\EEQA}{\end{eqnarray}}
\newcommand{\define}{\stackrel{\triangle}{=}}
\theoremstyle{remark}
\newtheorem{rem}{Remark}

% Marks the beginning of the document
\begin{document}
\bibliographystyle{IEEEtran}
\vspace{3cm}

\title{Learning {\LaTeX}}
\author{EE24BTECH11053 - S A Aravind Eswar$^{*}$}
\maketitle
\newpage
\bigskip

\renewcommand{\thefigure}{\theenumi}
\renewcommand{\thetable}{\theenumi}
\section{D MCQs with One or More than One Correct}
14. Let S be set of all column matrix $\myvec{
    b_1\\
    b_2\\
    b_3
}$ such that $b_1, b_2, b_3 \in \mathbb{R}$ and the system of equations (in real variables)\\ $
    \hfill -x+2y+5z=b_1\\
    \hfill 2x-4y+3z=b_2\\
    \hfill x-2y+2z=b_3
$\\ has at least one solution. Then, which of the following system(s) (in real variables) has (have) at least one solution for each $\myvec{
    b_1\\
    b_2\\
    b_3
}\in S$ \hfill(JEE Adv. 2018)

\begin{enumerate}
    \item $x+2y+3z=b_1, 4y+5z=b_2 \text{ and }x+2y+6z=b_3$
    \item $x+y+3z=b_1, 5x+2y+6z=b_2\text{ and }-2x-y-3z=b_3$
    \item $-x+2y-5z=b_1,2x-4y+10z=b_2\text{ and }x-2y+5z=b_3$
    \item $sx+2y+5z=b_1,2x+3z=b_2,x+4y-5z=b_3$\\[2pt]
\end{enumerate}

15. Let $M=\myvec{
    0 & 1 & a\\
    1 & 2 & 3\\
    3 & b & 1
}$ and $(\mathop{adj}M)=\myvec{
    -1 & 1 & -1\\
    8 & -6 & 2\\
    -5 & 3 & 1
}$ where $a$ and $b$ are real numbers. Which of the following options is/are correct? \hfill (JEE Adv. 2019)

\begin{enumerate}
    \item $a+b=3$
    \item $\mathop{det}(\mathop{adj}M^2)=81$
    \item $(adjM)^{-1}+adjM^{-1}=-M$
    \item $If M\myvec{
        \alpha\\
        \beta\\
        \gamma
} = \myvec{
        1\\
        2\\
        3
}$ then $\alpha-\beta+\gamma=3$\\[2pt]
\end{enumerate}

16. Let\\
$
P_1=I=\myvec{
    1 & 0 & 0\\
    0 & 1 & 0\\
    0 & 0 & 1
}, P_2 = \myvec{
    1 & 0 & 0\\
    0 & 0 & 1\\
    0 & 1 & 1
}, P_3 = \myvec{
    0 & 1 & 0\\
    1 & 0 & 0\\
    0 & 0 & 1
}, P_4 = \myvec{
    0 & 1 & 0\\
    0 & 0 & 1\\
    1 & 0 & 0
}, P_5 = \myvec{
    0 & 0 & 1\\
    1 & 0 & 0\\
    0 & 1 & 0
}, P_6 = \myvec{
    0 & 0 & 1\\
    0 & 1 & 0\\
    1 & 0 & 1
}
$ and $\displaystyle X = \sum_{k=1}^{6}P_k\myvec{
    2 & 1 & 3\\
    1 & 0 & 2\\
    3 & 2 & 1
}{P_k}^T$\\Where ${P_k}^T$ denotes the transpose of matrix $P_k$. Then which of the following options is/are correct? \hfill (JEE Adv. 2019)

\begin{enumerate}
	\item X is a symmetric matrix
	\item The sum of diagonal elements of X is 18
	\item X-30$I$ is an invertible matrix
	\item If $X\myvec{1\\1\\1}=\alpha\myvec{1\\1\\1},$ then a is 30\\[2pt]
\end{enumerate}

17. Let $x\in R$ and let
$$P = \myvec{1&1&1\\0&2&2\\0&0&3},Q = \myvec{2&x&x\\0&4&0\\x&x&5} \text{ and }R=PQP^{-1}$$
Then which of the following options is/are correct? \hfill (JEE Adv. 2019)


\begin{enumerate}
\item det $R = \text{det}\myvec{2&x&x\\0&4&0\\x&x&5}+8,$ for all $x\in R$
\item For $x=1$, there exists a unit vector $\alpha\hat{i}+\beta\hat{j}+\gamma\hat{k}$ for which $R\myvec{\alpha\\\beta\\\gamma}=\myvec{0\\0\\0}$
		\item There exists a real number $x$ such that $PQ = QP$
		\item For $x=0$, if $R=\myvec{1\\a\\b}=6\myvec{1\\a\\b},$ then a+b=5\\[2pt]
\end{enumerate}

\section{E Subjective Problems}

1. For what value of $k$ do the following system of equations possess a non trivial (i.e., not all zero) solution over the set of rationals $Q$?\\
$x+ky+3z=0\\3x+ky-2z=0\\2x+3y-4z=0$\\ For what value of k, find all the solutions of the system. \hfill (1979)\\[2pt]

2. Let $a,b,c$ be positive and not all equal.Show that the value of the determinant $\mydet{a&b&c\\b&c&a\\c&a&b}$ is negative. \hfill (1981 - 4 Marks)\\[2pt]

3. Wihout expanding a determinant at any stage, show that $\mydet{x^2+x&x+1&x-2\\2x^2+3x-1&3x&3x-3\\x^2+2x+3&2x-1&2x-1}=xA+B$, where $A$ and $B$ are determinants of order 3 not involving $x$. \hfill (1982 - 5 Marks)\\[2pt]

4. Show that $$\mydet{\Comb[x]{r}&\Comb[x]{r+1}&\Comb[x]{r+2}\\\Comb[y]{r}&\Comb[y]{r+1}&\Comb[y]{r+2}\\\Comb[z]{r}&\Comb[z]{r+1}&\Comb[z]{r+2}}=\mydet{\Comb[x]{r}&\Comb[x+1]{r+1}&\Comb[x+2]{r+2}\\\Comb[y]{r}&\Comb[y+1]{r+1}&\Comb[y+2]{r+2}\\\Comb[z]{r}&\Comb[z+1]{r+1}&\Comb[z+2]{r+2}}$$ \hfill (1985 - 2 Marks)\\[2pt]

5. Consider the system of linear equations the system of linear equations in x, y, z:\\$(\sin) 3\theta x-y+z=0\\(\cos 2\theta)x+4y+3z=0\\2x+7y+7z=0\\$ Find the values of $\theta$ for which this system has non trivial solutions. \hfill (1986 - 5 Marks)\\[2pt]

6. Let $\delta a=\mydet{a-1&n&6\\(a-1)^2&2n^2&4n-2\\(a-1)^3&3n^3&3n^2-3n}\\$Show that $\displaystyle\sum_{a=1}^{n}\Delta a=c$, a constant \hfill (1989-5 Marks)\\[2pt]

7. Let the three digit numbers $A28, 3B9,$ and $62C$, where $A, B,$ and $C$ are integers between 0 and 9, be divisible by a fixed integer k. Show that the determinant $\mydet{A&3&2\\8&9&c\\2&B&2}$ is divisible by $k$. \hfill (1990 - 4 Marks)\\[2pt]

8. If $a\neq p, b\neq q, c\neq r$ and $\mydet{p&b&c\\a&q&c\\a&b&r}=0$. Then find the value of $\frac{p}{p-a}+\frac{q}{q-b}+\frac{r}{r-c}$ \hfill (1991 - 4 Marks)\\[2pt]

9. For a fixed positive integer $n$, if $$D=\mydet{n!&(n+1)!&)(n+2)!\\(n+1)!&(n+2)!&(n+3)!\\(n+2)!&(n+3)!&(n+4)!}$$ then show that $\sbrak{\frac{D}{(n!)^3}-4}$ is divisible by $n$.\\[2pt]

10. Let $\lambda$ and $\alpha$ be real. Find the set of all values of $\lambda$ for which the system of linear equations $$\lambda x+(\sin\alpha)y+(\cos\alpha)=0, x+(\cos\alpha)y+(\sin\alpha)z=0,-x+(\sin\alpha)y-(\cos\alpha)z=0$$ has a non trivial solution. For $\lambda = 1$, find all values of $\alpha$. \hfill (1993 - 4 Marks)\\[2pt]

11. For all values of $A,B,C$ and $P,Q,R$ show that $$\mydet{\cos(A-P)&\cos(A-Q)&\cos(A-R)\\\cos(B-P)&\cos(B-Q)&\cos(B-R)\\\cos(C-P)&\cos(C-Q)&\cos(C-R)}=0$$ \hfill (1994 - 4 Marks)\\[2pt]

\end{document}

